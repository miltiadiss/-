\documentclass{report}
\usepackage{cite}
\bibliographystyle{IEEEtran}
\usepackage{array}
\usepackage{graphicx} 
\usepackage{titlepic}
\usepackage{wrapfig}
\usepackage[english,greek]{babel}
\usepackage[utf8x]{inputenc}
\usepackage{wrapfig}


% Greek-specific commands
\usepackage[english,greek]{babel}


\title{
\textbf{Εκμάθηση και Διδασκαλία Προγραμματισμού:\\ 
O Ρόλος της Τεχνητής Νοημοσύνης}
\author{Καραγιάννης Γεώργιος ΑΜ:1084586 \\Κολάγκη Ευαγγελία  ΑΜ:1084599 \\Μαντές Μηλτιάδης ΑΜ:1084661} }
\date{\LARGE \session Ιούνιος 2023}
\titlepic{\includegraphics[width=8cm]{uni.png}} 
\begin{document}
\maketitle

\tableofcontents
\selectlanguage{greek}
\chapter{Υπόμνημα(559)}
\selectlanguage{greek}
\section{Επιλογή Άρθρων}
Η τριμελής ομάδα μας για την εργασία στο μάθημα Συγγραφή και Παρουσίαση Τεχνικών Κειμένων, μετά από μία σύντομη μελέτη των δύο συλλογών άρθρων, συλλογικά αποφάσισε να επιλεχθεί η συλλογή \selectlanguage{english}Support Learning –Programming, \selectlanguage{greek}λόγω του μεγαλύτερου ενδιαφέροντός μας για το γνωστικό της πεδίο. Στην συνέχεια για να καταλήξουμε στα τελικά έξι (6) άρθρα που θα μελετηθούν χρησιμοποιήσαμε τον ιστότοπο: 
\selectlanguage{english} https://www.fileformat.info/tool/hash.htm. 
\selectlanguage{greek} \\Στο πρώτο πεδίο, \selectlanguage{english}String hash, \selectlanguage{greek} στο κελί \selectlanguage{english}Text \selectlanguage{greek}εισάγαμε αλφαβητικά τα ονοματεπώνυμα μας όπως παρακάτω: \\
\selectlanguage{english}\\Text:\selectlanguage{greek} ΚΑΡΑΓΙΑΝΝΗΣ ΚΟΛΑΓΚΗ ΜΑΝΤΕΣ\\ 
\\Υπολογίζονται με αυτόν τον τρόπο διάφορες συναρτήσεις κατακερματισμού, μοναδικά, με βάση τα ονόματα που εισάγαμε. Από τις υπολογισθείσες διαλέγουμε την \selectlanguage{english}SHA-1 (Secure Hash Algorithm 1) \selectlanguage{greek}που παράγει με την παραπάνω είσοδο έναν \selectlanguage{english}20-byte\selectlanguage{greek} ή 40-ψήφιο 16δικό αριθμό. Η δική μας μοναδική \selectlanguage{english}SHA-1 hash function \selectlanguage{greek}είναι η ακόλουθη:\\
\selectlanguage{english} \\ SHA-1: fe4613285d86883fb8a302d999285d1691dd7cff
\selectlanguage{greek}\\
\\Επιλέγουμε στην συνέχεια τα 6 πρώτα ΔΙΑΚΡΙΤΑ ψηφία του παραπάνω αριθμού, δοθέντος ότι δεν υπάρχουν στην συλλογή άρθρα με αριθμό \selectlanguage{english}f \selectlanguage{greek}ή \selectlanguage{english}e.\selectlanguage{greek}Τα ψηφία αυτά θα μας δώσουν τα τελικά 6(έξι) άρθρα που θα μελετήσουμε. \\
\\Διακριτά ψηφία:  461328 \\\\
\\\\\\Επομένως καταλήγουμε στα:\\\\
\selectlanguage{english}\textbf4 Learning and Teaching Programming: A Review and Discussion  \\\\
\textbf6  Here We Go Again: Why Is It Difficult for Developers to Learn Another Programming Language?  \\ \\
\textbf1  Conversing with Copilot: Exploring Prompt Engineering for Solving CS1 Problems Using Natural Language  \\\\ 
\textbf3  IS ABSTRACTION THE KEY TO COMPUTING?  \\\\ 
\textbf2  A Study of the Difficulties of Novice Programmers  \\\\
\textbf8  My AI Wants to Know if This Will Be on the Exam: Testing OpenAI’s Codex on CS2 Programming Exercises  \\
\selectlanguage{greek}

\section{Εξαίρεση Άρθρων}
Έπειτα από ανάλυση και μελέτη των άρθρων, αποφασίσαμε εξαιρέσουμε το άρθρο 3,\selectlanguage{english} \textit{IS ABSTRACTION THE KEY
TO COMPUTING?}\selectlanguage{greek}, καθώς αναφέρεται στην δεξιότητα της αφαίρεσης ως μιας δεξιότητας κλειδί για τους φοιτητές που ασχολούνται με την επιστήμη των υπολογιστών και ένα σοβαρό εργαλείο για πολλούς ανθρώπους διαφορετικών επαγγελμάτων. Επομένως, επειδή έχει την λιγότερο άμεση σχέση με τα υπόλοιπα άρθα και για να μην υπερβούμε στον λόγο μας, επιλέξαμε να μην το συμπεριλάβουμε καθώς δεν θεωρούμε ότι δίνει μια σημαντική προσθήκη στην βιβλιογραφική μας παρουσίαση.

\section{Ομάδα} 
Ως τριμελής ομάδα, κύριος σκοπός μας ήταν η ισότιμη κατανομή αρμοδιοτήτων σε όλα τα μέλη και η υλοποίηση τους εντός του χρονικού περιθωρίου. Για την ευκολότερη διαπεραίωση της εργασίας δημιουργήσαμε ένα χρονοδιάγραμμα σύμφωνα με τις ανάγκες της που όλα τα μέλη τήρησαν. \\
Αρχικά, όσον αφορά την μελέτη των άρθρων, το κάθε μέλος ήταν υπεύθυνο να μελετήσει 2 από τα 6 άρθρα και να υπογραμμίσει τα σημαντικότερα μέρη του ώστε να είναι ευκολότερο για να υπόλοιπα μέλη να τα κατανοήσουν. Επιπλέον αρμοδιότητα μας ήταν ο καθένας να γράψει περιληπτικά σε κατά μέγιστο 70-80 λέξεις ένα κείμενο που να περιγράφει την βασική ιδέα των δύο (2) άρθρων που είχε αναλάβει να μελετήσει. Σε ομαδική συνάντηση συνθέσαμε αρχικά την βιβλιογραφική παρουσίαση των έξι (6) άρθρων, μελετώντας από κοινού ξανά τα σημαντικά μέρη τους και παράλληλα παραθέτοντας τις δικές μας προσωπικές σημειώσεις. Στην συνέχεια, έγινε η συζήτηση με το \selectlanguage{english}ChatGPT\selectlanguage{greek} όπου παραθέσαμε τις απαντήσεις του και σχολιάσαμε την εγκυρότητα του.\\
Για την σύνθεση της εργασίας έκανε ο καθένας μας μία προσωπική μελέτη του\selectlanguage{english} LaTeX, \selectlanguage{greek} καθώς ήταν ένα εργαλείο με το οποίο όλοι ήρθαμε σε επαφή για πρώτη φορά και δεν ήμασταν εξοικιωμένοι με αυτό. Έτσι, ήταν πιο εύκολο να συνθέσουμε την εργασία, αφού ο καθένας μπορούσε να συμβάλει στην σύνταξη της.
Εν συνεχεία η Ευαγγελία ανέλαβε να συντάξει την Βιβλιογραφική Παρουσίαση, ο Μηλτιάδης να συντάξει τη Περίληψη και με την βοήθεια του Γιώργου να δημιουργήσει την παρουσίαση στο κείμενο 6 με βάση τις σημειώσεις της Ευαγγελίας και ο Γιώργος να συμπεριλάβει τα βιογραφικά και τις φωτογραφίες, καθώς και την συζήτηση με το \selectlanguage{english}ChatGPT\selectlanguage{greek} στο \selectlanguage{english}LaTeX\selectlanguage{greek}.Τέλος, μελετήσαμε την παρουσίαση σε \selectlanguage{english}PowerPoint\selectlanguage{greek} και δημιουργήσαμε το βίντεο για την παρουσίαση του κειμένου που είχαμε επιλέξει.

\chapter{Περίληψη(187)}
\section{\selectlanguage{english}Abstract\selectlanguage{greek}}
Κάνοντας μια ανασκόπηση της Βιβλιογραφικής Παρουσίασης που παρατίθεται πιο κάτω, προκύπτει ότι η κεντρική ιδέα που πραγματεύονται τα έξι (6) άρθρα δεν είναι άλλη από τη θέση που κατέχει ο προγραμματισμός στα πλαίσια του σήμερα, καθώς και η προσπάθεια διδασκαλίας του τόσο από ανθρώπους όσο και από υπολογιστές. Πιο συγκεκριμένα, αναλύονται διάφορες πτυχές που πλαισιώνουν έναν προγραμματιστή γενικότερα, όπως οι δεξιότητες που πρέπει να διαθέτει, η προσέγγιση που έχει απέναντι στη καινούργια γνώση αλλά και η στάση του προς παλαιότερες γνώσεις και ιδέες. Επίσης, φαίνεται ότι κάποια από τα άρθρα εξειδικεύουν αυτά τα γνωρίσματα ανάλογα με την εμπειρία ή μη του εκάστοτε προγραμματιστή. Ακόμα, δίνεται έμφαση και στο τρόπο μεταλαμπάδευσης της γνώσης από τους διδάσκοντες προς τους μαθητευόμενους και για την ακρίβεια το πόσο δύναται αυτός να βελτιωθεί με βάση τις ανάγκες και τις δυσκολίες που αντιμετωπίζουν οι τελευταίοι. Κλείνοντας, δεν παραλείπεται φυσικά και η αναφορά στη συμβολή της Τεχνητής Νοημοσύνης πάνω στη διαδικασία της εκμάθησης γλωσσών προγραμματισμού, καθώς αναλύονται διάφορα νεοσύστατα εργαλεία AI που παρέχουν με σχετικά μεγάλη επιτυχία γρήγορες και έμπιστες λύσεις σε προβλήματα, αντικαθιστώντας έτσι ένα τμήμα της διδασκαλίας που προέρχεται από τον άνθρωπο.  

\chapter{Βιβλιογραφική Παρουσίαση(1096)}
\section{Παρουσίαση Άρθρων}
Τι είναι αυτό που διευκολύνει την εκμάθηση προγραμματισμού? Είναι η τεχνητή νοημοσύνη ή οι προσωπικές δεξιότητες που διαθέτει ο εκάστοτε προγραμματιστής? Δεν είναι λίγες, άλλωστε, οι φορές που οι προγραμματιστές αναγκάζονται να καταφύγουν σε ψηφιακά εργαλεία τα οποία λειτουργούν βοηθητικά πάνω στο έργο τους. \\Σύμφωνα με τo άρθρο οκτώ(\textbf{8}) , ένα από αυτά αποτελεί το \selectlanguage{english}Codex\selectlanguage{greek}, ένα εργαλείο ΑΙ το οποίο εντοπίζει λύσεις σε προγραμματιστικά προβλήματα σε διάφορες γλώσσες προγραμματισμού. Εντύπωση προκαλεί το γεγονός ότι κατάφερε να επιλύσει ένα πρόβλημα καλύτερα δομημένο και από τους ίδιους τους φοιτητές. Αν και ισχυρό και διαθέσιμο τόσο για τους φοιτητές όσο και για τους δασκάλους, θα πρέπει να χρησιμοποιηθεί με σύνεση ώστε να μην επηρεαστεί η συμπεριφορά των φοιτητών αλλά και τις πρακτικές διδασκαλίας. \\Ένα ακόμη εργαλείο ΑΙ για το οποίο γίνεται λόγος στο άρθρο ένα(\textbf{1}) είναι το \selectlanguage{english}Capilot\selectlanguage{greek}, το οποίο δεχόμενο ως είσοδο ένα προγραμματιστικό πρόβλημα σε φυσική γλώσσα παράγει κώδικα για την λύση του. Παρατηρήθηκε ότι λύνει επιτυχώς με την πρώτη προσπάθεια περίπου τα μισά προβλήματα ενώ το 60\% των εναπομεινάντων με κάποιες αλλαγές στην φυσική γλώσσα περιγραφής του προβλήματος. Κατανοούμε λοιπόν ότι η τεχνητή νοημοσύνη δεν είναι απλώς ένα πιθανό βοήθημα αλλά μία, σε μεγάλο βαθμό, επιτυχημένη απόπειρα επίλυσης ενός προγραμματιστικού προβλήματος. \\
Ωστόσο, υπάρχουν και από την άλλη πλευρά δυσκολίες που δεν αντιμετωπίζονται εύκολα με κάποιο εργαλείο ΑΙ και αποτελούν τροχοπέδη για τους προγραμματιστές. Όπως διατυπώνεται στο άρθρο έξι(\textbf{6}), κατά την εκμάθηση μιας νέας γλώσσας προγραμματισμού, οι προγραμματιστές επιχειρούν από μόνοι τους να μάθουν μία νέα γλώσσα με στρατηγικές που συχνά συγκρίνουν προηγούμενες γλώσσες προγραμματισμού μεταξύ τους, πράγμα το οποίο δρα επιζήμια πάνω στην διαδικασία εκμάθησης.Αποδεδειγμένα, πάνω από τους μισούς κάνουν λανθασμένες υποθέσεις για μια γλώσσα προγραμματισμού, η οποία τους είναι παντελώς "ξένη", λόγω αλληλεπίδρασης με προηγούμενη γλώσσα που ήδη γνωρίζουν. Αιτία αποτελεί η δυσκολία να εγκαταλείψουν μια παλιά συνήθεια αλλά και τον γενικότερο τρόπο σκέψης που επιτάσσει μια γλώσσα προγραμματισμού, καθώς επίσης και η ανεπιτυχής προσαρμογή σε έναν νέο περιβάλλον συγγραφής κώδικα. Απαιτείται λοιπόν η αλλαγή της νοοτροπίας ότι ερχόμενος με παλαιές γνώσεις στον προγραμματισμό μαθαίνει με ευκολία οποιαδήποτε γλώσσα.  \\
Τα παραπάνω ωστόσο δεν περιορίζουν την διαφορά μεταξύ «αρχάριων» και «έμπειρων» προγραμματιστών, καθώς η διδασκαλία προγραμματισμού αποτελεί πιο περίπλοκο ζήτημα.Το άρθρο τέσσαρα(\textbf{4}) επικεντρώνεται στα ζητήματα που προβληματίζουν τους νέους προγραμματιστές και μελετά τα κύρια γνωρίσματα που χαρακτηρίζουν τη συμπεριφορά των αρχάριων σε σύγκριση με τους ειδικούς.Τονίζεται ότι τα στοιχεία που διακρίνουν έναν εξειδικευμένο από έναν μαθητευόμενο είναι η οργάνωση, η ταχύτητα, το εύρος γνώσεων και η ευελιξία. Ένας έμπειρος προγραμματιστής διαθέτει ικανότητες πρακτικές, καθώς διαθέτει το υπόβαθρο για να κατασκευάσει σχεδιαγράμματα και να διασπάσει το πρόβλημα σε απλούστερα υποπροβλήματα. Από την άλλη, ένας αρχάριος διαθέτει περισσότερο επιφανειακές γνώσεις πάνω στο αντικείμενο και έτσι τείνει να εφαρμόζει μη αποτελεσματικές στρατηγικές επίλυσης. Ένα ακόμη διαχωρισμός είναι ανάμεσα στους μαθητευόμενους προγραμματιστές.Πιο συγκεκριμένα, παρουσιάζονται δύο είδη αυτών: εκείνοι οι οποίοι μόλις συναντήσουν κάποιο εμπόδιο απογοητεύονται οπότε τα παρατούν και εκείνοι οι οποίοι αντιμετωπίζουν τα λάθη τους ως μια ευκαιρία για να πειραματισμό και να τροποποίηση μέχρι που φτάσουν σε λύση.\\ Στο άρθρο δύο(\textbf{2}) και στην παρουσίαση δυσκολιών των φοιτητών πληροφορικής και των καθηγητών τους, βρέθηκε ότι το μεγαλύτερο πρόβλημα που απασχολεί τους φοιτητές αποτελεί η μετάβαση τους από την θεωρία στην εφαρμογή. Αποδεικνύεται επίσης ότι οι φοιτητές προτιμούν την διδασκαλία και τη μελέτη στο σπίτι, υπερεκτιμώντας τις ικανότητες τους και θεωρούν ότι η μελέτη χωρίς την εποπτεία καθηγητή είναι πιο αποδοτική. Αντίθετα οι καθηγητές θεωρούν ότι με την αξιολόγηση εντοπίζουν τυχόν δυσκολίες που μπορεί οι φοιτητές να παραλείπουν. 

\section{\textbf{Κριτική συζήτηση με το} \selectlanguage{english}ChatGPT\selectlanguage{greek}}

Για την κριτηκή συζήτηση με το \selectlanguage{english}ChatGPT\selectlanguage{greek} επιλέξαμε το 2ο από τα άρθρα, δηλαδή το άρθρο 6\selectlanguage{english} \textit{Here We Go Again: Why Is It Difficult for Developers to Learn Another Programming Language? }\selectlanguage{greek}
Παρακάτω παραθέτουμε τις ερωτήσεις που θέσαμε και τις αντίστοιχες απαντήσεις που παρήγαγε.\\\\
\textbf{Ερώτηση 1}\\\\
\textbf{Εγώ}: Δώσε μια σύντομη περίληψη σχετικά με το άρθρο \selectlanguage{english} \textit{Here We Go Again: Why Is It Difficult for Developers to Learn Another Programming Language? By Nischal Shrestha, Colton Botta, Titus Barik, and Chris Parnin }\selectlanguage{greek} \\\\
\textbf{\selectlanguage{english}Chat\selectlanguage{greek}}: Το άρθρο αναλύει τους λόγους για τους οποίους οι προγραμματιστές δυσκολεύονται να μάθουν μια νέα γλώσσα προγραμματισμού, παρά την εμπειρία τους σε άλλες γλώσσες. Μετά από μια έρευνα πάνω σε προγραμματιστές που μετέβαλλαν από τη γλώσσα \selectlanguage{english}Java \selectlanguage{greek}στη \selectlanguage{english}Ruby\selectlanguage{greek}, οι συγγραφείς αναφέρουν ότι οι προγραμματιστές συνήθως αντιμετωπίζουν δυσκολίες κατά την εκμάθηση μιας νέας γλώσσας λόγω των διαφορών στη σύνταξη, τα παραδείγματα και τις βιβλιοθήκες, καθώς και της έλλειψης ευκαιριών εξάσκησης στη νέα γλώσσα. Επιπλέον, το άρθρο προτείνει ορισμένες πρακτικές για να βοηθήσει τους προγραμματιστές να μάθουν μια νέα γλώσσα πιο εύκολα και αποτελεσματικά.

Συγκρίνοντας την απάντηση που έδωσε το τσατ με το άρθρο που έχουμε ήδη διαβάσει, εύλογα συμπεράναμε ότι περιγράφει με ακρίβεια την κεντρική του ιδέα.\\\\

\textbf{Ερώτηση 2}\\

\textbf{Εγώ}: Συμφωνείς με το άρθρο και την δυσκολία που αντιμετωπίζουν οι προγραμματιστές όταν μαθαίνουν μία νέα γλώσσα;\\\\
\selectlanguage{english}\textbf{Chat}\selectlanguage{greek}: 
Ως Τεχνητή Νοημοσύνη δεν έχω προσωπική άποψη, αλλά μπορώ να παρατηρήσω ότι η δυσκολία στην μάθηση μιας νέας γλώσσας προγραμματισμού είναι κάτι που αντιμετωπίζουν πολλοί προγραμματιστές. Οι προγραμματιστές μπορεί να είναι εξοικειωμένοι με μια συγκεκριμένη γλώσσα προγραμματισμού και να δυσκολεύονται να προσαρμοστούν στα νέα στοιχεία που παρουσιάζονται στην νέα γλώσσα. Επιπλέον, η έλλειψη κατάλληλων εκπαιδευτικών πόρων και της κατάλληλης πρακτικής άσκησης μπορεί να καθιστά τη διαδικασία ακόμα πιο δύσκολη. Συνολικά, η μάθηση μιας νέας γλώσσας προγραμματισμού απαιτεί αφοσίωση και προσπάθεια, αλλά μπορεί να είναι επωφελής για την επαγγελματική ανάπτυξη του προγραμματιστή.\\\\
\textbf{Σχολιασμός}\\\\
Στην παραπάνω ερώτηση εύκολα κατανοεί κανείς ότι το \selectlanguage{english}ChatGPT\selectlanguage{greek} δεν εκφράζει μια συγκεκριμένη άποψη, αλλα δίνει μία άποψη αποπροσωποποιημένη και αντικειμενική πάνω στην διαδικασία εκμάθησης μίας γλώσσας προγραμματισμού. 
Ακόμη και ως προΐόν του ανθρώπινου νου, επιβεβαιώνουμε ότι δεν παύει να αποτελεί έναν αλγόριθμο που δεν έχει κριτική σκέψη αλλα προσπαθεί να συντάξει μία απάντηση συλλέγοντας "μηχανικά" δεδομένα σύμφωνα με την δική μας περιγραφή και να μιμειθεί μία ανθρωπογενή συμπεριφορά.\\\\

\textbf{Ερώτηση 3}\\\\
\textbf{Εγώ}:Πιστεύεις ότι μπορούν τα συστήματα τεχνητής νοημοσύνης να βοηθήσουν κάποιον στην εκμάθηση νέας γλώσσας προγραμματισμού;\\\\
\selectlanguage{english}\textbf{Chat}\selectlanguage{greek}: 
Ναι, τα συστήματα τεχνητής νοημοσύνης μπορούν να βοηθήσουν στην εκμάθηση νέων γλωσσών προγραμματισμού. Με τη χρήση τεχνικών μηχανικής μάθησης, τα συστήματα αυτά μπορούν να παρέχουν εξατομικευμένα μαθήματα και συμβουλές στους χρήστες με βάση την επίδοσή τους και τις ανάγκες τους.

Επιπλέον, η τεχνητή νοημοσύνη μπορεί να χρησιμοποιηθεί για τη δημιουργία εργαλείων εκμάθησης, όπως \selectlanguage{english}chatbots \selectlanguage{greek}, που μπορούν να απαντούν σε ερωτήσεις και να παρέχουν βοήθεια στους χρήστες. Επιπλέον, οι τεχνολογίες φωνητικής αναγνώρισης και ανάλυσης φυσικής γλώσσας μπορούν να χρησιμοποιηθούν για τη δημιουργία εφαρμογών που επιτρέπουν στους χρήστες να αλληλεπιδρούν με το σύστημα μέσω φωνητικών εντολών και φυσικής γλώσσας.

Ωστόσο, είναι σημαντικό να σημειωθεί ότι η τεχνητή νοημοσύνη δεν μπορεί να αντικαταστήσει πλήρως την ανθρώπινη διδασκαλία και την εμπειρία στην εκμάθηση νέων γλωσσών προγραμματισμού.\\
\textbf{\\Σχολιασμός}\\\\
Αναφορικά με το κείμενο και την απάντηση που δόθηκε παραπάνω καταληγουμε στο συμπέρασμα ότι τόσο ο συγγραφέας του άρθρου όσο και το \selectlanguage{english}chatbot \selectlanguage{greek} συγκλίνουν. Ναι μεν η Τεχνητή Νοημοσύνη μπορεί να συμβάλει στην διαδικασία εκμάθησης νεων γλωσσών, δεν παύει ωστόσο η ανθρώπινη παρουσία να διαδραματίζει σημαντικό ρόλο στην εκπαίδευση του ανθρώπου. 


\section{\textbf{Συζήτηση}}
Συνοψίζοντας, τα άρθρα που μας δόθηκαν θα πρέπει και θα ήταν καλό να συζητηθούν τόσο από φοιτητές όσο και από καθηγητές στις αίθουσες των πανεπιστημίων. Πιο συγκεκριμένα, τα άρθρα αυτά μπορούν να αποτελέσουν πηγή μάθησης και στοχασμού σχετικά με το πώς εκλαμβάνουν οι φοιτητές τις υποχρεώσεις που πρέπει να φέρουν εις πέρας κατά την διάρκεια των σπουδών τους. Επίσης μπορεί να βοηθήσει τους καθηγητές ώστε να έχουν μια ευρεία και μεγάλη γκάμα σχετικά με τα εργαλεία που μπορούν να δώσουν στους φοιτητές. Τέλος μπορούμε να καταλήξουμε στο συμπέρασμα πως είναι απαραίτητο να συζητηθούν διάφορα άρθρα στις πανεπιστημιακές σχολές προκειμένου οι φοιτητές και οι καθηγητές να έρθουν πιο κοντά στο αντικείμενο και την επιστήμη την οποία σπουδάζουν.

\selectlanguage{english}
\bibliography{bibliography}
{\selectlanguage{english}
\printbibliography(\textbf{282 Words}) \cite{denny2023conversing}, \cite{lahtinen2005study}, \cite{kramer2007abstraction}, \cite{robins2003learning}, \cite{tshukudu2020understanding}, \cite{shrestha2020here}, \cite{strijbos2004designing}, \cite{finnie2023my},\cite{dillenbourg1999you}, \cite{dillenbourg2010technology}}

\selectlanguage{greek}
\chapter{Βιογραφικά(341)}

\begin{wrapfigure}{r}{0.25\textwidth} %this figure will be at the right
    \centering
    \includegraphics[width=0.25\textwidth]{Karagiannis.png}
\end{wrapfigure}
Ο Γεώργιος Καραγιάννης είναι φοιτητής του τμήματος Μηχανικών Η/Υ & Πληροφορικής του πανεπιστήμιου Πατρών, όπου βελτιώνει τόσο τις τεχνικές όσο και τις  διαπροσωπικές του δεξιότητες. Αν και βρίσκεται στα πρώτα στάδια του κόσμου της μηχανικής και του προγραμματισμού, του αρέσει να εμπλέκετε και να συμμετάσχει σε διάφορα πρότζεκτ που αφορούν το κομμάτι των σπουδών του με ευχαρίστηση και δίχως κανένα δισταγμό. Μέσα σε αυτά τα τρία χρόνια έχει μάθει πώς να βλέπει δύσκολα και περίπλοκα προβλήματα με μια διαφορετική οπτική γωνία, βοηθώντας τον έτσι στο να καλλιεργήσει την κριτική του σκέψη. Τέλος το χαρακτηριστικό του που ξεχωρίζει είναι οι ισχυρές επικοινωνιακές του δεξιότητες, οι οποίες τον κάνουν έναν πολύτιμο «παίκτη» τόσο σε ένα εργασιακό περιβάλλον, όσο και σε φιλικό.

\vspace{10mm}
\begin{wrapfigure}{l}{0.25\textwidth} %this figure will be at the right
    \centering
    \includegraphics[width=0.20\textwidth]{20230310_160026_(1)_cleanup}
\end{wrapfigure}
Η Κολάγκη Ευαγγελία, φοιτήτρια του τμήματος Μηχανικών Ηλεκτρονικών Υπολογιστών και
πληροφορικής του πανεπιστημίου Πατρών, έχοντας ολοκληρώσει επιτυχώς τα
μέχρι τώρα μαθήματα του προγράμματος σπουδών, έχει σχηματίσει βασικές
γνώσεις σχετικά με θέματα θεμελίωσης της πληροφορικής τόσο με λογικό όσο και
σε υλικό επίπεδο. Αναζητεί συνεχώς νέες γνώσεις και εμπειρίες, όντας ενεργό
μέλος σε εθελοντικές δράσεις, όπως βοηθός επικουρικού έργου στο εργαστήριο
Ψηφιακών ηλεκτρονικών και εθελοντής μέλος της επιστημονικής ομάδας IEEE του
πανεπιστημίου Πατρών. Είναι αφοσιωμένη και με υψηλά κίνητρα, με βασικό της
χαρακτηριστικό την ομαδικότητα και την οργάνωση. Έπειτα από τρία χρόνια
ακαδημαϊκής εμπειρίας, τα κέντρα ενδιαφέροντος της είναι η Επιστήμη
Δεδομένων και ο Προγραμματισμός.
\vspace{5mm}

 \begin{wrapfigure}{l}{0.25\textwidth} %this figure will be at the right
    \centering
    \includegraphics[width=0.20\textwidth]{345214729_113814598351594_565123789024838587_n.jpg}
\end{wrapfigure}
Ο Μηλτιάδης Μαντές ανήκει στους προπτυχιακούς φοιτητές του Τμήματος ΜΗΥΠ του
Πανεπιστημίου Πατρών(2020 – ). Κατά τη διάρκεια της φοίτησής του στο τμήμα έχει
αποκτήσει σημαντική εμπειρία πάνω σε θεμελιώδη κομμάτια της επιστήμης των Η/Υ.
Τομείς όπως το Λογικό, οι Θεμελιώσεις και οι Εφαρμογές της Επιστήμης των
Υπολογιστών αποτελούν κάποια από τα κύρια ενδιαφέροντά του, ενώ παράλληλα
επόμενος στόχος του είναι να διευρύνει τις γνώσεις του πάνω σε εξειδικευμένα θέματα
Ευρυζωνικών Τεχνολογιών και Ασφάλειας Δικτύων. Επιπλέον, έχει ασχοληθεί
εκτεταμένα με ζητήματα Προγραμματισμού και Διαχείρισης Βάσεων Δεδομένων.
Όσον αφορά την εθελοντική δράση, στο τρέχον εξάμηνο προσφέρει Επικουρικό Έργο
για το Εργαστήριο Ψηφιακών Ηλεκτρονικών. Τέλος, το ομαδικό πνεύμα, ο σωστός
χρονοπρογραμματισμός και η προσαρμοστικότητα είναι κοινωνικές δεξιότητες που
τον χαρακτηρίζουν άμεσα.



\selectlanguage{greek}
{\chapter{Παρουσίαση}}
Παρακάτω παραθέτουμε το \selectlanguage{english} link \selectlanguage{greek} για την παρουσίαση του άρθρου έξι(6):\\  \selectlanguage{english}
https://docs.google.com/presentation/d/1swgzDjWIRvzpDpr52WV259WTsmP-uwIK/edit?usp=share\textunderscore link\&ouid=118127185583468130436&rtpof=true&sd=true
\end{document}
